\documentclass[12pt,a4paper]{article}
\usepackage{etex,datetime,setspace,latexsym,amssymb,amsmath,amsthm}
\usepackage{fancybox,dialogue,float,wrapfig,enumerate,microtype}
\usepackage{verbatim,xcolor,multicol,titlesec,tabularx,mdframed}

\usepackage[utf8]{inputenc}
\usepackage[pdftex]{hyperref}
\usepackage[margin=3cm,bottom=3cm,footskip=15mm]{geometry}
\parindent0cm
\parskip1em

\usepackage{tikz}
\usetikzlibrary{arrows,trees,positioning,shapes,patterns}
\usetikzlibrary{intersections,calc,fpu,decorations.pathreplacing}

% \usepackage[T1]{fontenc} % better fonts

% Haskell code listings in our own style
\usepackage{listings,color}
\definecolor{lightgrey}{gray}{0.35}
\definecolor{darkgrey}{gray}{0.20}
\definecolor{lightestyellow}{rgb}{1,1,0.92}
\definecolor{dkgreen}{rgb}{0,.2,0}
\definecolor{dkblue}{rgb}{0,0,.2}
\definecolor{dkyellow}{cmyk}{0,0,.7,.5}
\definecolor{lightgrey}{gray}{0.4}
\definecolor{gray}{gray}{0.50}
\lstset{
  language        = Haskell,
  basicstyle      = \scriptsize\ttfamily,
  keywordstyle    = \color{dkblue},     stringstyle     = \color{red},
  identifierstyle = \color{dkgreen},    commentstyle    = \color{gray},
  showspaces      = false,              showstringspaces= false,
  rulecolor       = \color{gray},       showtabs        = false,
  tabsize         = 8,                  breaklines      = true,
  xleftmargin     = 8pt,                xrightmargin    = 8pt,
  frame           = single,             stepnumber      = 1,
  aboveskip       = 2pt plus 1pt,
  belowskip       = 8pt plus 3pt
}
\lstnewenvironment{code}[0]{}{}
\lstnewenvironment{showCode}[0]{\lstset{numbers=none}}{} % only shown, not compiled

% will the real phi please stand up
\renewcommand{\phi}{\varphi}

% load hyperref as late as possible for compatibility
\usepackage[pdftex]{hyperref}
\hypersetup{pdfborder = {0 0 0}, breaklinks = true}


\title{Sudoku Solving in Haskell}
\author{Robin Martinot, 10589155}
\date{\today}
\hypersetup{pdfauthor={Robin}, pdftitle={SudokuSolving}}

\begin{document}

\maketitle

\begin{abstract}
The goal of this project was to adapt and extend existing code for solving Sudoku
puzzles in Haskell to code that solves and generates various different types of Sudokus. Additionally, the difficulty of solving these types of Sudokus was measured
by implementing a counter in a recursive function of the code. It seems that with the
current measure of complexity it is hard to conclude exactly how `difficult' Haskell finds
the Sudoku, although the new types of Sudokus seemed harder to solve than the regular Sudoku.
Nevertheless, the results were rather variable and additional research is required in order
to draw any reliable conclusions.
\end{abstract}

\section{Introduction}
Implementation of numeric puzzles, in particular Sudokus, can be
carried out in many programming languages. Haskell, too, has been used to generate and solve
the logic-based number-placement Sudoku puzzle (for existing code, see https://homepages.cwi.nl/~jve/courses/16/fsa/lab/FSAlab3.html).
Briefly, the objective of the Sudoku is to fill a $9 \times 9$ grid with digits so that each column, each row, and each
of the nine $3 \times 3$ subgrids that compose the grid contain all of the digits
from 1 to 9. Translating this to code, this entails that there has to
be a surjective and injective function $f: \{1, ..., 9 \} \rightarrow R$ where $R$ is
one of the rows, one of the columns or one of the subgrids.\\\\
It is unclear whether code adapted to variations of this puzzle changes
the complexity of finding a solution such that it is detectable in Haskell. We might expect that adding
extra requirements to the puzzle forces Haskell to check for consistency more often, so that
it is less easy to find suitable values. Alternatively, additional requirements might limit the possibilities
for new values in open positions, so that it is in fact easier for Haskell to extend the Sudoku with new values.\\\\
To test these hypotheses, code was generated in the current project for two different types of Sudokus besides the regular one.
A `diagonal' Sudoku required both diagonals to contain all of the values $1$ up to $9$. Additionally,
a Sudoku type was implemented with four additional $3 \times 3$ blocks with left-top corners
$(2,2)$, $(2,6)$, $(6,2)$, and $(6,6)$ (coordinates denoting positions by (row, column)).
An attempt at comparing the complexity of the different types of Sudokus was made.

\section{Methods}
(The complete code can be found at the bottom of this report). First, the code for regular Sudokus was adapted to `diagonal' Sudokus and Sudokus
with additional subgrids. This entailed adding a definition of the required additional blocks, and
adding a `Restriction' type which specifies the elements of both diagonals and the additional blocks
(besides row, column and subgrid).
Instead of writing code for each restriction separately, the Restriction type was added as input
argument to functions like `injectiveFor', `consistent' and `freeAtPos', so that these only needed one instance.
For example, the latter function, which computes which values are available at a certain position in the grid, now looks like this:
\begin{code}
freeAtPos :: [Restriction] -> Sudoku -> (Row,Column) -> [Value]
freeAtPos x s (r,c) = foldr1 intersect [ freeInSeq (allValues res s (r,c)) | res <- x ]
\end{code}
The following lists of restrictions were defined, which describe all of the requirements
that each type of Sudoku should adhere to:
\begin{code}
normal :: [Restriction]
normal = [row, column, subGrid]

diagonal :: [Restriction]
diagonal = [row, column, subGrid, diagonalUp, diagonalDown]

addGrid :: [Restriction]
addGrid = [row, column, subGrid, addsubGrid]
\end{code}
In particular, the solveAndShow function now works by taking into account the type of Sudoku:
\begin{code}
solveAndShow :: Grid -> [Restriction] -> IO ()
solveAndShow gr x = solveShowNs x (initNode gr x)
\end{code}

Second, the random Sudoku generator was adapted so that it generates each type of
Sudoku. For this, an additional input argument for the list of restrictions that
decides the type of Sudoku was added to the existing functions. This way,
the code generates a filled Sudoku of the right type and subsequently removes values
until no value can be removed without the Sudoku becoming unsolvable.\\
The generated problem Sudoku is then fed back into the Sudoku solver. For this, the function
`genAndSolve' was defined:
\begin{code}
genAndSolve :: [Restriction] -> IO ()
genAndSolve x = do (_,s) <- sudGen x
                   showSudoku s
                   solveAndShow (sud2grid s) x
\end{code}\\\\
Third, to gain more insight into the search process, a counter was added to the recursive function `search'.
This way, the number of tries Haskell needs to find a solution was kept track of.
The function `solveAndShow' returns the number of counted steps along
with the solved Sudoku.\\\\
\begin{code}
search :: Int -> (node -> [node])
        -> (node -> Bool) -> [node] -> ([node], Int)
search w _ _ [] = ([],w)
search w children goal (x:xs)
  | goal x    = (x : fst (search w children goal xs), w)
  | otherwise = search (w+1) children goal (children x ++ xs)
\end{code}
Finally, the `genAndSolve' function was run 20 times for each type of Sudoku.
\begin{code}
genAndSolve :: [Restriction] -> IO ()
genAndSolve x = do (_,s) <- sudGen x
                  showSudoku s
                  solveAndShow (sud2grid s) x
\end{code}

\section{Results}
The results are displayed in Table 1, 2 and 3 below. Each column shows the results of
running genAndSolve ten times. The average number of steps per column can be found in bold at the bottom,
below which the overall average is shown.
\begin{center}
 \begin{tabular}{|c | c |}
 \hline
 \multicolumn{2}{|c|}{\textbf{Number of steps regular Sudoku}} \\[0.5ex]
 \hline\hline
 62 & 248  \\
 \hline
 133 & 76 \\
 \hline
 59 & 594 \\
 \hline
 248 & 290 \\
 \hline
 232 & 75 \\
 \hline
 319 & 473 \\
 \hline
 55 & 57 \\
 \hline
 59 & 60 \\
 \hline
 60 & 96 \\
 \hline
 419 & 976 \\
 \hline\hline
 \textbf{164.6} & \textbf{294.5} \\
 \hline
 \multicolumn{2}{|c|}{229.55} \\
 \hline
\end{tabular}
\end{center}
\noindent { \small $\it{Table {}1}$. This shows the number of steps Haskell
needed to solve the regular Sudoku 20 times.}

\begin{center}
 \begin{tabular}{|c | c |}
 \hline
 \multicolumn{2}{|c|}{\textbf{Number of steps addGrid Sudoku}} \\[0.5ex]
 \hline\hline
 238 & 30343  \\
 \hline
 1598 & 355681 \\
 \hline
 839 & 3279 \\
 \hline
 1544 & 770 \\
 \hline
 1548 & 13369 \\
 \hline
 938 & 17853 \\
 \hline
 496 & 9234 \\
 \hline
 1247 & 7838 \\
 \hline
 23577 & 1452 \\
 \hline
 1604 & 7444 \\
 \hline\hline
 \textbf{3362.9} & \textbf{44726.3} \\
 \hline
 \multicolumn{2}{|c|}{24044.6} \\
 \hline
\end{tabular}
\end{center}
\noindent { \small $\it{Table {}2}$. This shows the number of steps Haskell
needed to solve the Sudoku with additional grids 20 times.}

\begin{center}
 \begin{tabular}{|c | c |}
 \hline
 \multicolumn{2}{|c|}{\textbf{Number of steps diagonal Sudoku}} \\[0.5ex]
 \hline\hline
 290 & 264  \\
 \hline
 4376 & 146 \\
 \hline
 343 & 1053 \\
 \hline
 1311 & 1105 \\
 \hline
 9870 & 26144 \\
 \hline
 548 & 1926 \\
 \hline
 977 & 1455 \\
 \hline
 182 & 220 \\
 \hline
 385 & 4136 \\
 \hline
 2295 & 928 \\
 \hline\hline
 \textbf{2057.7} & \textbf{3737.7} \\
 \hline
 \multicolumn{2}{|c|}{2897.7} \\
 \hline
\end{tabular}
\end{center}
\noindent { \small $\it{Table {}3}$. This shows the number of steps Haskell
needed to solve the diagonal Sudoku 20 times.}

We see that the regular Sudoku on average requires the least number of steps (229.55 steps). Next
in line is the diagonal Sudoku with 2897.7 steps on average. The Sudoku with four
additional subgrids required most steps on average (24044.6 steps). However, there was
one extreme result for this type of Sudoku (355681 steps) which considerably affected this
average. Without this particular result, the average for this type of Sudoku is 6590.053 steps instead of
24044.6. Even though the new average is still higher than that of the diagonal Sudoku,
this shows that the results are subject to this type of fluctuations.

\section{Discussion \& conclusion}
In short, it appears that the results are relatively inconsistent, although a pattern
can be detected. The regular Sudoku seems to require the least number of steps that
Haskell needs, followed by the diagonal Sudoku. The Sudoku with four additional blocks
requires by far most steps. In line with these results, the approximate computation time
for Haskell for diagonal Sudokus and Sudokus with additional blocks exceeded that of the regular
Sudokus.\\\\
The results might be explained by the fact that normal Sudokus require less checking
for Haskell, as they have the least number of requirements that need to be fulfilled.
Furthermore, checking for additional subgrids each overlapped with four normal types of grids,
perhaps causing the computation to become more complex. The diagonals, although seemingly similar to additional rows
or columns, overlap with three normal subgrids and also need to fulfill $r + c == 10$ and $r == c$ for each of
their elements. Apparently adding these constraints makes a difference in complexity when solving the Sudoku. Thus, a preliminary conclusion
is that diagonal Sudokus and Sudokus with additional grids are indeed harder to solve
for Haskell.\\\\
In the current project it was assumed that every randomly generated Sudoku was
approximately of the same difficulty. This is not straightforward, however: the random Sudoku generator sometimes may have happened to generate
a rather easily solvable Sudoku, and other times a hard one. The number of Sudokus that
was currently run (20 of each type) is not enough to take away possible effects caused by this (e.g. if it happened to generate easier Sudokus
for the normal type in this experiment). In fact, as mentioned
in the results section this seems to have been of influence on the averages. Thus, the
results cannot yet be taken as reliable.\\\\
It should also be kept in mind that there is no obvious and official way to measure the
complexity of a Sudoku. The current method was only one way of analyzing the differences
between the solving of different types of Sudokus. To give a more complete view of
how Haskell solves different Sudokus, measures of other functions, computation time measurements and
visualization of intermediate steps in the solving process can be taken into account.
To be more precise it is necessary for there to be specific rules for what we define an easy or hard
Sudoku problem.\\\\
Something I did not have time to complete in the current project is to compare the
results to binary puzzles. These are Sudoku-like puzzles that only work with values
$0,1$ and require slightly different rules. Additionally, there are many other types
of Sudokus that can still be investigated in Haskell, as well as the mentioned different measures
of complexity.\\\\
Below, the entire code of the project is displayed, with comments explaining what
changes to the existing code were made.


\clearpage

% We include one file for each section. The ones containing code should
% be called something.lhs and also mentioned in the .cabal file.

\input{AllSudokusNew.lhs}

\addcontentsline{toc}{section}{Bibliography}
\bibliographystyle{alpha}
\bibliography{references.bib}
Code for regular Sudokus provided by Jan van Eijck, https://homepages.cwi.nl/~jve/courses/16/fsa/lab/FSAlab3.html

\end{document}
